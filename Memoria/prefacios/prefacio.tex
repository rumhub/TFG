\chapter*{}
%\thispagestyle{empty}
%\cleardoublepage

%\thispagestyle{empty}

\begin{titlepage}
 
 
\setlength{\centeroffset}{-0.5\oddsidemargin}
\addtolength{\centeroffset}{0.5\evensidemargin}
\thispagestyle{empty}

\noindent\hspace*{\centeroffset}\begin{minipage}{\textwidth}

\centering
%\includegraphics[width=0.9\textwidth]{imagenes/logo_ugr.jpg}\\[1.4cm]

%\textsc{ \Large PROYECTO FIN DE CARRERA\\[0.2cm]}
%\textsc{ INGENIERÍA EN INFORMÁTICA}\\[1cm]
% Upper part of the page
% 

 \vspace{3.3cm}

%si el proyecto tiene logo poner aquí
\includegraphics[width=0.9\textwidth]{imagenes/logo_ugr.jpg}\\[1.4cm]
 \vspace{0.5cm}

% Title

{\LARGE \bfseries Implementación optimizada sobre sistemas heterogéneos de algoritmos de Deep Learning para clasificación de imágenes\\
}
\noindent\rule[-1ex]{\textwidth}{3pt}\\[3.5ex]
\end{minipage}

\vspace{2.5cm}
\noindent\hspace*{\centeroffset}\begin{minipage}{\textwidth}
\centering

\textbf{Autor}\\ {David Sánchez Pérez}\\[2.5ex]
\textbf{Directores}\\
{José Miguel Mantas Ruiz}\\[2cm]
%\includegraphics[width=0.15\textwidth]{imagenes/tstc.png}\\[0.1cm]
%\textsc{Departamento de Teoría de la Señal, Telemática y Comunicaciones}\\
%\textsc{---}\\
%Granada, mes de 201
\end{minipage}
%\addtolength{\textwidth}{\centeroffset}
\vspace{\stretch{2}}

 
\end{titlepage}






\cleardoublepage
\thispagestyle{empty}

\begin{center}
{\large\bfseries Implementación optimizada sobre sistemas heterogéneos de algoritmos de Deep Learning para clasificación de imágenes}\\
\end{center}
\begin{center}
David Sánchez Pérez\\
\end{center}

%\vspace{0.7cm}
\noindent{\textbf{Palabras clave}: Inteligencia Artificial, Aprendizaje Automático, Aprendizaje Profundo, Arquitectura de Redes Neuronales, Procesamiento de Imágenes, Algoritmos de Aprendizaje Automático, Implementación en C++, Paralelización con OpenMP, Miltihilo, Computación en GPU, Programación CUDA, NVIDIA CUDA, Biblioteca cuDNN, Implementaciones Personalizadas}\\

\vspace{0.7cm}
\noindent{\textbf{Resumen}}\\

Este \textit{Trabajo de Fin de Grado} se centra en la aplicación de la computación heterogénea a bajo nivel para optimizar el rendimiento de redes neuronales convolucionales (CNNs). Las CNNs son esenciales en una variedad de campos, incluyendo el reconocimiento de imágenes, la visión artificial, la detección de objetos y el procesamiento del lenguaje natural. 

El enfoque del proyecto es diseñar y desarrollar diversas implementaciones para optimizar la combinación de CPUs y GPUs en algoritmos de aprendizaje profundo. En primer lugar, se diseñará la arquitectura de las redes neuronales convolucionales a implementar. A continuación, se llevará a cabo una implementación secuencial en CPU, seguida de una implementación paralela en CPU utilizando OpenMP. Posteriormente, se desarrollará un sistema heterogéneo empleando CUDA, y, finalmente, se implementará otro sistema heterogéneo utilizando la biblioteca cuDNN.

Cada implementación será construida desde cero, sin depender de bibliotecas o frameworks externos, salvo por OpenMP, CUDA y cuDNN, con el objetivo de consolidar los conocimientos sobre CNNs y su optimización en sistemas heterogéneos.
\cleardoublepage


\thispagestyle{empty}


\begin{center}
{\large\bfseries Optimized Implementation of Deep Learning Algorithms for Image Classification on Heterogeneous Systems}\\
\end{center}
\begin{center}
David Sánchez Pérez\\
\end{center}

%\vspace{0.7cm}
\noindent{\textbf{Keywords}: Artificial Intelligence, Machine Learning, Deep Learning, Computer Vision, Neural Network Architecture, Image Processing, Machine Learning Algorithms, C++ Implementation, OpenMP Parallelization, Miltithreading, GPU Computing, CUDA Programming, NVIDIA CUDA, cuDNN Library, Custom Implementations}\\

\vspace{0.7cm}
\noindent{\textbf{Abstract}}\\

This \textit{Bachelor's Thesis} focuses on the application of low-level heterogeneous computing to optimize the performance of convolutional neural networks (CNNs). CNNs are crucial in various fields, including image recognition, computer vision, object detection, and natural language processing.

The project's approach is to design and develop several implementations to enhance the synergy between CPUs and GPUs in deep learning algorithms. First, the architecture of the convolutional neural networks to be implemented will be designed. Next, a sequential implementation on CPU will be carried out, followed by a parallel CPU implementation using OpenMP. Subsequently, a heterogeneous system using CUDA will be developed, and finally, another heterogeneous system utilizing the cuDNN library will be implemented.

Each implementation will be built from scratch, without relying on external libraries or frameworks, except for OpenMP, CUDA, and cuDNN, with the aim of consolidating knowledge about CNNs and their optimization on heterogeneous systems.

\chapter*{}
\thispagestyle{empty}

\noindent\rule[-1ex]{\textwidth}{2pt}\\[4.5ex]

Yo, \textbf{Nombre Apellido1 Apellido2}, alumno de la titulación TITULACIÓN de la \textbf{Escuela Técnica Superior
de Ingenierías Informática y de Telecomunicación de la Universidad de Granada}, con DNI XXXXXXXXX, autorizo la
ubicación de la siguiente copia de mi Trabajo Fin de Grado en la biblioteca del centro para que pueda ser
consultada por las personas que lo deseen.

\vspace{6cm}

\noindent Fdo: Nombre Apellido1 Apellido2

\vspace{2cm}

\begin{flushright}
Granada a X de mes de 201 .
\end{flushright}


\chapter*{}
\thispagestyle{empty}

\noindent\rule[-1ex]{\textwidth}{2pt}\\[4.5ex]

D. \textbf{Nombre Apellido1 Apellido2 (tutor1)}, Profesor del Área de XXXX del Departamento YYYY de la Universidad de Granada.

\vspace{0.5cm}

D. \textbf{Nombre Apellido1 Apellido2 (tutor2)}, Profesor del Área de XXXX del Departamento YYYY de la Universidad de Granada.


\vspace{0.5cm}

\textbf{Informan:}

\vspace{0.5cm}

Que el presente trabajo, titulado \textit{\textbf{Título del proyecto, Subtítulo del proyecto}},
ha sido realizado bajo su supervisión por \textbf{Nombre Apellido1 Apellido2 (alumno)}, y autorizamos la defensa de dicho trabajo ante el tribunal
que corresponda.

\vspace{0.5cm}

Y para que conste, expiden y firman el presente informe en Granada a X de mes de 201 .

\vspace{1cm}

\textbf{Los directores:}

\vspace{5cm}

\noindent \textbf{Nombre Apellido1 Apellido2 (tutor1) \ \ \ \ \ Nombre Apellido1 Apellido2 (tutor2)}

\chapter*{Agradecimientos}
\thispagestyle{empty}

       \vspace{1cm}


Poner aquí agradecimientos...

