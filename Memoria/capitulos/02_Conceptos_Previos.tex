\chapter{Conceptos previos}


\section{Machine Learning}

Se entiende como el campo de las ciencias de computación que en vez de enfocarse en el diseño de algoritmos explícitos, optan por el estudio de técnicas de aprendizaje. Este enfoque tiene un gran éxito en tareas computacionales donde no es factible diseñar un algoritmo de forma explícita. \cite{Programming_Massively} \\
En vez de averiguar las distintas reglas a seguir para llegar a una solución, esta alternativa permite simplemente suministrar ejemplos de lo que debería pasar en distintas situaciones, y dejar que la máquina aprenda y extraiga ella misma sus propias conclusiones. De esta forma, el procedimiento en aprendizaje supervisado consiste en 'entrenar' con una muestra de N ejemplos, extraer información de ellos, y posteriormente poder evaluar de forma 'correcta' (bajo un margen de error controlado) otra muestra de M ejemplos, siendo M \textgreater N. \cite{Learning_From_Data} \\
Este enfoque ha contribuido en el avance de áreas como reconocimiento de voz, visión por ordenador, procesamiento de lenguaje natural, etc.

\section{Componentes necesarios para el aprendizaje supervisado}

Datos de entrada X y de salida Y que el modelo empleará para aprender y tomar decisiones. Ambos se unen para formar un dataset de entradas-salidas D=\{($x_1, y_1$), ($x_2, y_2$), ..., ($x_N, y_N$)\}. Para que el aprendizaje sea posible, debe existir una función F: X $\rightarrow$ Y tal que $y_i$ = F($x_i$) para i$\in$\{1...N\}. De esta forma, en función del dataset D, el modelo tratará de encontrar una función G que aproxime F para dicho conjunto. Además, se suelen aplicar técnicas que permitan una mejor generalización del modelo, expandiendo las capacidades del mismo y permitiendo que su conocimiento pueda ser útil incluso fuera de la muestra de datos inicial. \cite{Learning_From_Data}

\section{División de datos en entrenamiento y test}

Para visualizar la generalización del modelo, el conjunto de datos D se suele dividir en 2 subconjuntos, (entrenamiento y test) de forma que se pueda estimar si realmente 'aprende' o solo memoriza.\\
Una vez realizada la división, se entrena el modelo con los datos del conjunto de entrenamiento. Cuando se termina el entrenamiento, se accede al conjunto test y se visualiza el rendimiento del modelo sobre el mismo. Como los datos de test no se emplearon en ningún momento, aportan una estimación sobre la generalización del modelo fuera de la muestra con la que se entrenó. 

\section{Tipos de aprendizaje}

\subsection{Aprendizaje Supervisado}

% \cite{Learning_From_Data} página 24

Es el que se empleará en este proyecto. 
Se caracteriza por la presencia de una etiqueta 'correcta' $y_i$ asociada a cada dato de entrada $x_i$. Posteriormente, la red empleará ambos valores para, a partir de $x_i$, tratar de deducir $y_i$. \cite{Learning_From_Data} \\
Aunque se trate de impedir, la existencia de ruido en los datos es inevitable, implicando que algunas etiquetas de Y=$\{y_1, y_2, ..., y_N\}$ puedan ser erróneas. \\
Este tipo de aprendizaje se divide a su vez en problemas de clasificación y regresión, centrándose en predecir etiquetas o valores numéricos, respectivamente.

\subsection{Aprendizaje No Supervisado}

En este tipo de aprendizaje los datos no contienen ninguna información respecto a Y. De esta forma, el conjunto de datos D se compone exclusivamente de valores X=$\{x_1, x_2, ..., x_N\}$. \cite{Learning_From_Data}

\subsection{Aprendizaje Por Refuerzo}

En este caso tampoco existe un $y_i$ 'correcto' asociado a cada $x_i$. En su lugar, se asocia a cada $x_i$ una etiqueta con un valor posible de $y_i$, además de una medida que indica como de bueno es el mismo. \cite{Learning_From_Data}


\section{Redes Neuronales Totalmente Conectadas}

\subsection{Neurona}

\begin{figure}[H]
	\centering
	\includegraphics[scale=0.35]{imagenes/neurona.jpg}  
	\caption{Imagen de una neurona}
	\label{fig:neurona}
\end{figure}

Una neurona parte de una serie de datos de entrada X=\{$x_1$, $x_2$, ..., $x_N$\} tal que cada $x_i$$\in${X} se encuentra asociado a un peso $w_i\in{W}$. \\
Esta los emplea para realizar una suma ponderada y posteriormente añadir un sesgo b, además de aplicar una función de activación f sobre el resultado obtenido. 

\subsection{Estructura por capas}

\begin{figure}[H]
	\centering
	\includegraphics[scale=0.35]{imagenes/capa_neuronas.jpg}  
	\caption{Imagen de una capa de neuronas}
	\label{fig:capa_neuronas}
\end{figure}

Las neuronas se suelen agrupas por capas, de tal forma que la salida de una compone la entrada de la siguiente, formando así modelos más sofisticados.

\subsection{Funciones de activación}

\subsubsection{ReLU}

\begin{gather}
	ReLU(x) = max(0, x)
\end{gather}

\begin{figure}[H]
	\centering
	\includegraphics[scale=0.45]{imagenes/ReLU.jpg}  
	\caption{Imagen de la función de activación ReLU}
	\label{fig:ReLU}
\end{figure}

A cambio de un bajo coste computacional, aporta no linealidad a la neurona, permitiendo a esta aprender funciones de mayor complejidad. \\
Como su gradiente es 0 o 1, evita una reducción excesiva del mismo para valores positivos, mitigando así el problema del desvanecimiento del gradiente, caracterizado por la presencia de gradientes muy pequeños en backpropagation y provocar un aprendizaje lento. \cite{ReLU}

\subsubsection{Sigmoide}

\begin{gather}
	sigmoide(x) = \frac{1}{1+e^{-x}}
\end{gather}

\begin{figure}[H]
	\centering
	\includegraphics[scale=0.45]{imagenes/sigmoide.jpg}  
	\caption{Imagen de la función de activación Sigmoide}
	\label{fig:Sigmoide}
\end{figure}

Se trata de una función interesante en el ámbito de la clasificación binaria, pues se caracteriza por transformar un valor de entrada en una salida comprendida en el rango [0-1]. \\
Aunque sea monótona creciente y diferenciable en todos los puntos, tiende a saturarse con valores extremos (positivos o negativos). Por tanto, su aplicación dependerá del caso concreto a tratar. \cite{Sigmoide}

\subsubsection{SoftMax}

\begin{figure}[H]
	\centering
	\includegraphics[scale=0.35]{imagenes/softmax.jpg}  
	\caption{Imagen de la función de activación SoftMax}
	\label{fig:SoftMax}
\end{figure}

Para n entradas, produce n salidas con valores en el rango [0-1] que mantienen la proporción de entrada y cuya suma es 1. Por tanto, se pueden interpretar como la probabilidad de pertenencia a cada clase, siendo especialmente útil en clasificación multiclase. \cite{SoftMax_MLM} \\
\subsection{One-hot encoding}

\subsection{Función de error o pérdida}

\subsubsection{Entropía Cruzada}

\begin{gather}
	E(y, \hat{y}) = - \sum_{i=1}^{H}  [y_i * log( \hat{y}_i)]
	\label{loss_func_softmax}
\end{gather}

Es una métrica empleada en aprendizaje automático para medir qué tan bien se desempeña un modelo de clasificación. La pérdida o error se mide como un valor en el rango [0-1], siendo 0 un modelo perfecto y 1 otro totalmente erróneo. \cite{Cross_entropy}

H es el número de clases al que puede pertenecer cada dato de entrada $x_i \in X$.


% https://medium.com/mlearning-ai/understanding-loss-functions-for-classification-81c19ee72c2a

\subsubsection{Sigmoid Cross Entropy Loss}
Es un caso particular de entropía cruzada caracterizado por la presencia de un número de clases igual a 2.

\begin{gather}
	E(x) = - \frac{1}{N} \sum_{i=1}^{N}  [y_i * log( \hat{y}_i) + (1-y_i)*log(1-\hat{y_i})]
	\label{loss_func}
\end{gather}

y = etiqueta real \\
$\hat{y}$ = predicción 

\subsection{Descenso del gradiente}

\begin{figure}[H]
	\centering
	\includegraphics[scale=0.4]{imagenes/sgd.jpg}  
	\caption{Ejemplo de funcionamiento del descenso del gradiente}
	\label{fig:SGD}
\end{figure}

Es un método de optimización iterativo que busca el mínimo local en una función diferenciable. En la figura \ref{fig:SGD} se muestra un ejemplo del mismo, donde cada punto representa una iteración del algoritmo. \\
Su nombre viene del término 'gradiente', siendo este una generalización multivariable de la derivada y denominado por el símbolo $\nabla$. Para una función f y un punto p, este indica la dirección del máximo incremento en la misma. El descenso del gradiente usa esta información para, una vez obtenido el gradiente, desplazarse en dirección contraria, es decir, en dirección del mínimo. Además, la distancia que se recorre en cada iteración viene dada por un hiperparámetro denominado ``learning rate'' o $\alpha$ \cite{SGD_1} \cite{Gradiente} \cite{SGD_2}. \\


\subsubsection{Entrenamiento}
De esta forma, el procedimiento para entrenar una red neuronal consiste en, para una entrada $x_i$ y una etiqueta asociada $y_i$, emplear $x_i$ para producir una predicción $\hat{y}_i$ (\textbf{ForwardPropagation} o Propagación hacia delante) que posteriormente se podrá comparar con $y_i$ mediante una función de error H(x) y obtener una medida de lo buena o mala que fue la misma. Una vez obtenido dicho ``error'', se aplica el algoritmo del descenso del gradiente para cada parámetro de la red (\textbf{BackPropagation} o retropropagacion) \cite{Cross_entropy}. \\

\begin{gather}
	W_{t+1} = W_{t} - \alpha * \frac{\partial H(x)}{\partial W_{t}} 
	\label{act_pesos}
\end{gather}

\begin{gather}
	b_{t+1} = b_{t} - \alpha * \frac{\partial H(x)}{\partial b_{t}}
	\label{act_bias}
\end{gather}


Así, se actualizarán los parámetros de la red neuronal según las fórmulas \ref{act_pesos} y \ref{act_bias}. En ellas, $W_t$ y $b_t$ indican los valores del peso W y bias b en el instante o iteración t, de la misma forma que $W_{t+1}$ y $b_{t+1}$ representan los valores de los mismos en el instante t+1 \cite{SGD_act_params}.

\subsubsection{Descenso del gradiente estocástico}

\begin{algorithm}[H]
	\caption{Descenso del gradiente estocástico \cite{SGD_3}} 
	\begin{algorithmic}
		\State Datos de entrenamiento $D=\{(x_1, y_1), (x_2, y_2), ..., (x_N, y_N)\}$.
		\For{época $p\in\{0, ..., P-1\}$}
			\State Desordenar vector de datos D.
			
			\For{$i \in [m_{inicio}, ..., m_{fin}]$}
				\State Obtener pareja de datos $(x_i, y_i) \in D$
				\State Realizar propagación hacia delante de $x_i$ y obtener predicción $\hat{y}_i$.
				\State Evaluar $\hat{y}_i$ con la función de pérdida y calcular el error $e_i$.
				\State Realizar propagación hacia detrás y obtener gradientes 
				\State      de cada parámetro del modelo.
				\State Actualizar parámetros.
			\EndFor
		\EndFor
	\end{algorithmic}
\end{algorithm}

Es una variante que sustituye el gradiente real por una estimación del mismo, logrando reducir la carga computacional y tiempo de entrenamiento a cambio de una menor tasa de convergencia \cite{sgd_stocastico} \cite{sgd_stocastico_1}. \\
Se caracterizada por, en cada época, dividir el conjunto de entrenamiento en varios subconjuntos aleatorios y disjuntos entre ellos (mini-batch), de tal forma que se calcule el gradiente y actualicen los parámetros del modelo en cada uno de ellos \cite{sgd_stocastico}. \\

\subsection{Inicialización de pesos y sesgos}

\subsubsection{Inicialización de pesos}
Como función de activación se empleará ReLU. Por tanto, tal y como se indica en la bibliografía, se inicializan los pesos mediante la ``inicialización He'' o ``inicialización Kaiming He''. Esta consiste en, para un peso $w$, inicializarlo según una distribución gaussiana con una media de 0.0 y una desviación típica de $\sqrt{\frac{2}{N\_in}}$, siendo $N\_in$ el número de neuronas en la capa de entrada
 \cite{ini_He} \cite{ini_He_2} \cite{ini_He_code}.

\subsubsection{Inicialización de sesgos}

De la misma forma, se sigue la bilbiografía y los sesgos se inicializarán a 0.0 \cite{ini_bias} \cite{ini_bias_2}.

\subsection{Tipos de codificaciones}

En el campo de machine learning existen varios tipos de codificaciones. De esta forma, para codificar 3 clases distintas se podrían codificar o bien mediante \{1, 2, 3\} (codificación de etiquetas), o mediante \{100, 010, 001\} (codificación one-hot), por ejemplo. En este proyecto se empleará la codificación one-hot, pues aporta unas ventajas que se contemplarán con detalle en secciones posteriores.

\subsection{Propagación hacia delante con softmax}

Suponemos que para un $x_i$ dado, se obtiene S($\hat{y}$) = [ S($\hat{y}_1)$, S($\hat{y}_2)$, S($\hat{y}_3)$ ] = [0.04, 0.7, 0.26 ] \\
Para dicho $x_i$, $y_i$ = [0, 0, 1] \\
En este caso, el modelo cree que $x_i$ pertenece a la clase 2 (0.7 es el mayor número del vector S($\hat{y})$). Sin embargo, $y_i$ indica que $x_i$ pertenece a la clase 3. \\

Se calcula el error de entropía cruzada: \\

\begin{gather}
	 E(y, S(\hat{y})) = - (0*log(0.04) + 0*log(0.7) + 1*log(0.26)) \\
	 E(y, S(\hat{y})) = -log(0.26) = 0.585
\end{gather}

\cite{Cross_entropy_backprop}

\section{Redes Neuronales Convolucionales}

Las redes neuronales convolucionales (CNNs) se caracterizan por trabajar con volúmenes 3D de datos. En este proyecto, en ocasiones estos también podrán denominarse como imágenes, pues una imagen RGB se trata de 3 matrices 2D, una por color. Además, la entrada de cada capa se define como X o X(C*H*W) y en ambos casos cuenta con unas dimensiones de C*H*W, siendo C el número de canales de profundidad, H el número de filas y W el número de columnas. De la misma forma, se define el volumen de salida como Y($M*H_{out}*W_{out}$). En cuanto a los pesos de la capa, cada filtro o kernel se denominará mediante un símbolo/s diferente/s (K,K1, G, etc), pues cada uno puede presentar unas dimensiones diferentes. Aun así, se podrá definir a un filtro de pesos como K1(K*K), indicando que posee K filas y columnas. Además, también se trata de una estructura con 3 dimensiones. Sin embargo, no se requiere especificar el número de canales de profundidad de un kernel de pesos pues siempre será C (depende de X). De la misma forma, el número de filtros a aplicar sobre X recibe el nombre de M, y el tamaño de Y depende de este, pues tal y como se verá posteriormente, cada filtro genera una canal de profundidad distinto de la salida Y. 

\subsection{Capa convolucional}

\subsubsection{Componentes}


\begin{figure}[H]
	\centering
	\includegraphics[scale=0.35]{imagenes/conv_nombres.jpg}  
	\caption{Componentes en una capa convolucional}
	\label{fig:Componentes_convolucion}
\end{figure}

Una capa convolucional parte de un volumen de entrada X, un kernel de filtros K, un sesgo B y una función de activación para, mediante una convolución, obtener un volumen de salida Y \cite{capa_convolucional} \cite{capa_convolucional_Stanford}.

\subsubsection{Propagación hacia delante}

\begin{figure}[H]
	\centering
	\begin{subfigure}{.5\textwidth}
		\hspace{-10mm}
		\includegraphics[width=1.2\linewidth]{imagenes/conv_1.jpg}  
		\caption{Cálculo $Y_{11}$}
	\end{subfigure}%
	\begin{subfigure}{.5\textwidth}
		\hspace{10mm}
		\includegraphics[width=1.2\linewidth]{imagenes/conv_2.jpg}  
		\caption{Cálculo $Y_{12}$}
	\end{subfigure}
	
	\vspace{5mm}
	\begin{subfigure}{.5\textwidth}
		\hspace{-10mm}
		\includegraphics[width=1.2\linewidth]{imagenes/conv_3.jpg}  
		\caption{Cálculo $Y_{21}$}
	\end{subfigure}%
	\begin{subfigure}{.5\textwidth}
		\hspace{10mm}
		\includegraphics[width=1.2\linewidth]{imagenes/conv_4.jpg}  
		\caption{Cálculo $Y_{22}$}
	\end{subfigure}
	\caption{Propagación hacia delante en una capa convolucional}
	\label{fig:forward_prop_convolucional}
\end{figure}

Una convolución entre 2 volúmenes de datos X y K, consiste en ``deslizar K sobre X'' tal y como se muestra en la Figura \ref{fig:forward_prop_convolucional}. 
De esta forma, en cada ``paso'' se recorren ambos volúmenes, multiplicando los elementos de X y K 
que se encuentren en la misma posición. Posteriormente, se suma cada resultado obtenido, además de un sesgo B y finalmente aplicar una función de activación \cite{capa_convolucional}. \\
En la Figura \ref{fig:forward_prop_convolucional} se emplea un volumen X con un solo canal de profundidad. Sin embargo, este no es el caso común. Por tanto, se denotará como $X^{c}_{ij}$ al elemento de X que se encuentre en la posición (i,j) del canal de profundidad c \cite{capa_convolucional_Stanford}.

\subsubsection{Propagación hacia delante, canales de profundidad}

\begin{figure}[H]
	\centering
	\begin{subfigure}{.5\textwidth}
		\hspace{-10mm}
		\includegraphics[width=1.2\linewidth]{imagenes/conv_1_1canal.jpg}  
		\caption{Cálculo $Y_{11}$ con entrada X \\ de 1 canal de profundidad}
	\end{subfigure}%
	\begin{subfigure}{.5\textwidth}
		\hspace{10mm}
		\includegraphics[width=1.2\linewidth]{imagenes/conv_1_2canales.jpg}  
		\caption{Cálculo $Y_{11}$ con entrada X \\ de 2 canales de profundidad}
	\end{subfigure}
	
	\vspace{5mm}
	\begin{subfigure}{.5\textwidth}
		\hspace{-10mm}
		\includegraphics[width=1.2\linewidth]{imagenes/conv_1_3canales.jpg}  
		\caption{Cálculo $Y_{11}$ con entrada X \\ de 3 canales de profundidad}
	\end{subfigure}
	\caption{Propagación hacia delante en una capa convolucional con varios canales de profundidad}
	\label{fig:forward_prop_convolucional_canales_profundidad}
\end{figure}

De esta forma, en la Figura \ref{fig:forward_prop_convolucional_canales_profundidad} se muestra como una convolución con C canales de profundidad se descompone en la suma de C convoluciones con un canal de profundidad. \\ Para una entrada X con C canales de profundidad, convolución(X,K) = convolución($X^1,K^1$) + convolución($X^2,K^2$) + ... + convolución($X^C,K^C$). \\
Por último, en cada ``paso'' del ``deslizamiento'' se suma un solo sesgo y se aplica una sola vez la función de activación, independientemente del número de canales de profundidad que presente la entrada X.

\subsubsection{Propagación hacia delante, número de filtros}

\begin{figure}[H]
	\centering
	\begin{subfigure}{.5\textwidth}
		\hspace{-10mm}
		\includegraphics[width=1.2\linewidth]{imagenes/conv_2kernels_1.jpg}  
		\caption{Cálculo de $Y^1_{11}$ con el filtro K}
	\end{subfigure}%
	\begin{subfigure}{.5\textwidth}
		\hspace{10mm}
		\includegraphics[width=1.2\linewidth]{imagenes/conv_2kernels_2.jpg}  
		\caption{Cálculo $Y^2_{11}$ con el filtro G}
	\end{subfigure}
	\caption{Propagación hacia delante en una capa convolucional con varios filtros}
	\label{fig:forward_prop_convolucional_varios_kernels}
\end{figure}

Cada convolución entre dos volúmenes 3D produce un volumen de salida 2D. Por tanto, al aplicar N convoluciones entre un volumen de entrada X y una serie de filtros K=$\{K_1, K_2, ..., K_N\}$, se obtendrá un volumen 3D de salida con tantas capas de profundidad como convoluciones se aplicaron (N). \\
En la Figura \ref{fig:forward_prop_convolucional_varios_kernels} se observa como al aplicar N=2 convoluciones sobre la misma entrada X (una con el filtro K y otra con el filtro G) se obtiene un volumen de salida con 2 capas de profundidad \cite{capa_convolucional_Stanford}.

\subsubsection{Relleno o ``Padding''}

En la figura \ref{fig:forward_prop_convolucional} se visualiza como al realizar una convolución entre un volumen X con dimensiones 1x4x4 y un kernel de pesos K con dimensiones 1x3x3, el resultado obtenido es un volumen Y de dimensiones 1x2x2. La reducción de dimensionalidad es un problema pues afecta directamente al número de convoluciones que se pueden aplicar sobre un volumen. \\
Por tanto, el ``relleno' o ``padding'' se aplica antes de realizar una convolución y es una técnica empleada para conservar las dimensiones espaciales de un volumen de entrada X, expandiendo cada canal del mismo tal y como se muestra en la figura \ref{fig:padding} \cite{padding_1}.

\begin{figure}[H]
	\centering
	\begin{subfigure}{.5\textwidth}
		\hspace{-15mm}
		\includegraphics[width=1.2\linewidth]{imagenes/padding_a_1_nivel.jpg}  
		\caption{1 nivel de relleno}
	\end{subfigure}%
	\begin{subfigure}{.5\textwidth}
		\hspace{5mm}
		\includegraphics[width=1.2\linewidth]{imagenes/padding_a_2_niveles.jpg}  
		\caption{2 niveles de relleno}
	\end{subfigure}
	\caption{Relleno sobre un volumen de entrada X}
	\label{fig:padding}
\end{figure}

De esta forma, un relleno a un nivel sobre un volumen añadirá sobre el mismo dos filas y dos columnas con valores igual a 0. Un relleno a dos niveles añadirá cuatro filas y columnas con valores igual a cero, ..., un relleno a n niveles añadirá 2*n filas y columnas con valores igual a 0.

\subsubsection{Relleno completo}

Se denomina como relleno completo aquel que asegura que cada valor o casilla de X sea visitada el mismo número de veces que el resto \cite{padding_2}.

\begin{figure}[H]
	\centering
	\subfloat[1 nivel de relleno]{%
		\includegraphics[width=0.45\textwidth]{imagenes/conv_padding_1.jpg}%
		\label{fig:sub1}%
	}\hfill
	\subfloat[2 niveles de relleno]{%
		\includegraphics[width=0.45\textwidth]{imagenes/conv_padding_2.jpg}%
		\label{fig:sub2}%
	}\\
	\subfloat[1 nivel de relleno]{%
		\includegraphics[width=0.45\textwidth]{imagenes/conv_padding_3.jpg}%
		\label{fig:sub3}%
	}\hfill
	\subfloat[2 niveles de relleno]{%
		\includegraphics[width=0.45\textwidth]{imagenes/conv_padding_4.jpg}%
		\label{fig:sub4}%
	}\hfill
	
	\caption{Relleno sobre un volumen de entrada X}
\end{figure}

\begin{figure}[H]\ContinuedFloat
	\centering
	\subfloat[1 nivel de relleno]{%
		\includegraphics[width=0.45\textwidth]{imagenes/conv_padding_5.jpg}%
		\label{fig:sub5}%
	}\hfill
	\subfloat[2 niveles de relleno]{%
		\includegraphics[width=0.45\textwidth]{imagenes/conv_padding_6.jpg}%
		\label{fig:sub6}%
	}\\
	\subfloat[1 nivel de relleno]{%
		\includegraphics[width=0.45\textwidth]{imagenes/conv_padding_7.jpg}%
		\label{fig:sub7}%
	}\hfill
	\subfloat[2 niveles de relleno]{%
		\includegraphics[width=0.45\textwidth]{imagenes/conv_padding_8.jpg}%
		\label{fig:sub8}%
	}\hfill
		\subfloat[1 nivel de relleno]{%
		\includegraphics[width=0.45\textwidth]{imagenes/conv_padding_9.jpg}%
		\label{fig:sub9}%
	}\hfill
	\subfloat[2 niveles de relleno]{%
		\includegraphics[width=0.45\textwidth]{imagenes/conv_padding_10.jpg}%
		\label{fig:sub10}%
	}\hfill
		\subfloat[1 nivel de relleno]{%
		\includegraphics[width=0.45\textwidth]{imagenes/conv_padding_11.jpg}%
		\label{fig:sub11}%
	}\hfill
	\subfloat[2 niveles de relleno]{%
		\includegraphics[width=0.45\textwidth]{imagenes/conv_padding_12.jpg}%
		\label{fig:sub12}%
	}\hfill
	\subfloat[1 nivel de relleno]{%
		\includegraphics[width=0.45\textwidth]{imagenes/conv_padding_13.jpg}%
		\label{fig:sub13}%
	}\hfill
	\subfloat[2 niveles de relleno]{%
		\includegraphics[width=0.45\textwidth]{imagenes/conv_padding_14.jpg}%
		\label{fig:sub14}%
	}\hfill
	\subfloat[1 nivel de relleno]{%
		\includegraphics[width=0.45\textwidth]{imagenes/conv_padding_15.jpg}%
		\label{fig:sub15}%
	}\hfill
	\subfloat[2 niveles de relleno]{%
		\includegraphics[width=0.45\textwidth]{imagenes/conv_padding_16.jpg}%
		\label{fig:sub16}%
	}\\
	
	\caption{Relleno sobre un volumen de entrada X}
	\label{fig:full_padding}
\end{figure}

Tal y como se muestra en la figura \ref{fig:full_padding}, se realiza un relleno completo pues en la convolución entre X y K se accede a cada valor de X el mismo número de veces (9).

\subsection{Capa de agrupación máxima}

\subsubsection{Componentes}

\begin{figure}[H]
	\centering
	\includegraphics[scale=0.35]{imagenes/pool_nombres.jpg}  
	\caption{Componentes en una capa de agrupación máxima}
\end{figure}

Al igual que las capas convolucionales, las capas de agupación máxima también presentan una ``ventana'' que se irá deslizando por el volumen de entrada. Sin embargo, el resultado en cada iteración viene dado por el valor máximo contenido en ella. Por tanto, no presenta parámetros asociados.

\subsubsection{Propagación hacia delante}

\begin{figure}[H]
	\centering
	\begin{subfigure}{.5\textwidth}
		\hspace{-10mm}
		\includegraphics[width=1.2\linewidth]{imagenes/maxpool_1.jpg}  
		\caption{Cálculo $Y_{11}$}
	\end{subfigure}%
	\begin{subfigure}{.5\textwidth}
		\hspace{10mm}
		\includegraphics[width=1.2\linewidth]{imagenes/maxpool_2.jpg}  
		\caption{Cálculo $Y_{12}$}
	\end{subfigure}
	
	\vspace{5mm}
	\begin{subfigure}{.5\textwidth}
		\hspace{-10mm}
		\includegraphics[width=1.2\linewidth]{imagenes/maxpool_3.jpg}  
		\caption{Cálculo $Y_{21}$}
	\end{subfigure}%
	\begin{subfigure}{.5\textwidth}
		\hspace{10mm}
		\includegraphics[width=1.2\linewidth]{imagenes/maxpool_4.jpg}  
		\caption{Cálculo $Y_{22}$}
	\end{subfigure}
	\caption{Propagación hacia delante en una capa de agrupación máxima}
	\label{fig:forward_prop_maxpool}
\end{figure}

A diferencia de las capas convolucionales, las capas de agrupación máxima no comparten regiones del volumen de entrada entre distintas iteraciones. Esto se muestra en las figuras \ref{fig:forward_prop_maxpool} y \ref{fig:forward_prop_convolucional}.


\subsubsection{Propagación hacia delante, canales de profundidad}

\begin{figure}[H]
	\centering
	\begin{subfigure}{.5\textwidth}
		\hspace{-10mm}
		\includegraphics[width=1.2\linewidth]{imagenes/maxpool_2capas_1.jpg}  
		\caption{Cálculo $Y_{11}$}
	\end{subfigure}%
	\begin{subfigure}{.5\textwidth}
		\hspace{10mm}
		\includegraphics[width=1.2\linewidth]{imagenes/maxpool_2capas_2.jpg}  
		\caption{Cálculo $Y_{12}$}
	\end{subfigure}
	
	\caption{Propagación hacia delante en una capa de agrupación máxima}
	\label{fig:forward_prop_maxpool_canales_profundidad}
\end{figure}

Tal y como se muestra en la figura \ref{fig:forward_prop_maxpool_canales_profundidad}, cada ``ventana'' se desliza sobre el volumen de entrada en un canal de profundidad distinto.

\subsubsection{Propagación hacia detrás}

\begin{figure}[H]
	\centering
	\begin{subfigure}{.5\textwidth}
		\hspace{-10mm}
		\includegraphics[width=1.2\linewidth]{imagenes/back_maxpool_1.jpg}  
		\caption{Cálculo $Y_{11}$}
	\end{subfigure}%
	\begin{subfigure}{.5\textwidth}
		\hspace{10mm}
		\includegraphics[width=1.2\linewidth]{imagenes/back_maxpool_2.jpg}  
		\caption{Cálculo $Y_{12}$}
	\end{subfigure}
	
	\begin{subfigure}{.5\textwidth}
		\hspace{-10mm}
		\includegraphics[width=1.2\linewidth]{imagenes/back_maxpool_3.jpg}  
		\caption{Cálculo $Y_{11}$}
	\end{subfigure}%
	\begin{subfigure}{.5\textwidth}
		\hspace{10mm}
		\includegraphics[width=1.2\linewidth]{imagenes/back_maxpool_4.jpg}  
		\caption{Cálculo $Y_{12}$}
	\end{subfigure}%
	\caption{Propagación hacia delante en una capa de agrupación máxima}
	\label{fig:back_prop_maxpool}
\end{figure}

Para la primera iteración de la figura \ref{fig:back_prop_maxpool}, en la propagación hacia detrás se calcula el gradiente respecto al volumen de entrada de la siguiente forma:

\begin{gather}
	\frac{\partial E}{\partial X^1_{11}} = \frac{\partial E}{\partial Y} * \frac{\partial Y}{\partial X^1_{11}} \\
	\frac{\partial E}{\partial X^1_{12}} = \frac{\partial E}{\partial Y} * \frac{\partial Y}{\partial X^1_{12}} \\
	\frac{\partial E}{\partial X^1_{21}} = \frac{\partial E}{\partial Y} * \frac{\partial Y}{\partial X^1_{21}} \\
	\frac{\partial E}{\partial X^1_{22}} = \frac{\partial E}{\partial Y} * \frac{\partial Y}{\partial X^1_{22}}
\end{gather}

 
Tomando el ejemplo de la figura \ref{fig:back_prop_maxpool} (a), se obtiene que $Y^1_{11} = X^1_{11}$. Por tanto, las fórmulas anteriores se convierten en:

\begin{gather}
	\frac{\partial E}{\partial X^1_{11}} = \frac{\partial E}{\partial Y} * \frac{\partial X^1_{11}}{\partial X^1_{11}} = \frac{\partial E}{\partial Y} * 1 = \frac{\partial E}{\partial Y} \\
	\frac{\partial E}{\partial X^1_{12}} = \frac{\partial E}{\partial Y} * \frac{\partial X^1_{11}}{\partial X^1_{12}} = \frac{\partial E}{\partial X^1_{11}} * 0 \\
	\frac{\partial E}{\partial X^1_{21}} = \frac{\partial E}{\partial Y} * \frac{\partial X^1_{11}}{\partial X^1_{21}} = \frac{\partial E}{\partial X^1_{11}} * 0 \\
	\frac{\partial E}{\partial X^1_{22}} = \frac{\partial E}{\partial Y} * \frac{\partial X^1_{11}}{\partial X^1_{22}} = \frac{\partial E}{\partial X^1_{11}} * 0
\end{gather}


 
Como el resultado obtenido en cada iteración solo depende del valor máximo de la ventana, tiene sentido que la derivada de la salida Y respecto a la entrada X sea igual a 1 solo en dicho caso y 0 en el resto \cite{max_pool_backprop} \cite{max_pool_backprop_2}.
\subsection{Capa de aplanado}

\subsubsection{Propagación hacia delante}
\begin{figure}[H]
	\centering
	\begin{subfigure}{.5\textwidth}
		\hspace{-10mm}
		\includegraphics[width=2\linewidth]{imagenes/flatten_1.jpg}  
		\caption{Propagación hacia delante de la capa de aplanado con la primera capa de profundidad}
	\end{subfigure}
	\begin{subfigure}{.5\textwidth}
		\hspace{-10mm}
		\includegraphics[width=2\linewidth]{imagenes/flatten_2.jpg}  
		\caption{Propagación hacia delante de la capa de aplanado con la segunda capa de profundidad}
	\end{subfigure}
	
	\caption{Propagación hacia delante en una capa de agrupación máxima}
	\label{fig:forward_prop_flatten_canales_profundidad}
\end{figure}

La capa de aplanado tiene como objetivo la creación de un ``enlace'' entre las capas de convolución y agrupación con la capas totalmente conectadas. En la propagación hacia delante, parte de un volumen de entrada 3D para, mediante un ``aplanado'', convertirlo en un vector 1D que pueda ser usado como entrada para una red totalmente conectada. \\
Como solo se modifica la forma en la que se agrupan los datos, no requiere la presencia de parámetros \cite{flatten_forward}.

\subsubsection{Propagación hacia detrás}

\begin{figure}[H]
	\centering
	\begin{subfigure}{.5\textwidth}
		\hspace{-30mm}
		\includegraphics[width=2\linewidth]{imagenes/back_flatten_1.jpg}  
		\caption{Propagación hacia detrás de la capa de aplanado con la primera capa de profundidad}
	\end{subfigure}
	\begin{subfigure}{.5\textwidth}
		\hspace{-30mm}
		\includegraphics[width=2\linewidth]{imagenes/back_flatten_2.jpg}  
		\caption{Propagación hacia detrás de la capa de aplanado con la segunda capa de profundidad}
	\end{subfigure}
	
	\caption{Propagación hacia delante en una capa de agrupación máxima}
	\label{fig:back_prop_flatten_canales_profundidad}
\end{figure}

En la propagación hacia detrás, se parte de un array 1D y se convierte en un volumen 3D que pueda ser usado como entrada para una capa convolucional o de agrupación
\cite{flatten_forward}.
