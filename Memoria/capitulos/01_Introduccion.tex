\chapter{Introduccion}

\section{Machine Learning}

Se entiende como el campo de las ciencias de computación que en vez de enfocarse en el diseño de algoritmos explícitos, optan por el estudio de técnicas de aprendizaje. Este enfoque tiene un gran éxito en tareas computacionales donde no es factible diseñar un algoritmo de forma explícita. \cite{Programming_Massively} \\
En vez de averiguar las distintas reglas a seguir para llegar a una solución, esta alternativa permite simplemente suministrar ejemplos de lo que debería pasar en distintas situaciones, y dejar que la máquina aprenda y extraiga ella misma sus propias conclusiones. De esta forma, el procedimiento en aprendizaje supervisado consiste en 'entrenar' con una muestra de N ejemplos, extraer información de ellos, y posteriormente poder evaluar de forma 'correcta' (bajo un margen de error controlado) otra muestra de M ejemplos, siendo M \textgreater N. \cite{Learning_From_Data} \\
Este enfoque ha contribuido en el avance de áreas como reconocimiento de voz, visión por ordenador, procesamiento de lenguaje natural, etc.


\section{Deep Learning}

\section{Tipos de aprendizaje}


\subsection{Aprendizaje Supervisado}



\subsection{Aprendizaje No Supervisado}


\subsection{Aprendizaje Por Refuerzo}


\section{Tipos de problemas en machine learning}


\section{División de datos en entrenamiento y test}



\section{Entrenamiento}


